\chapter{Adicionando a rede}
Redes Legacy envolve principalmente nós de computação. No entanto, é necessário configurar o nó controlador para usar a rede Legacy

\section{Configurando o nó controlador}
\begin{enumerate}
	\item Editar o arquivo \emph{/etc/nova/nova.conf} e completar as seguintes ações
	\begin{snugshade}
		[default]\\
		$\cdots$ \\
		network$\_$api$\_$class = nova.network.api.API \\
		security$\_$group$\_$api = nova
	\end{snugshade}
	
	\item Reiniciar os serviços Compute
	\begin{snugshade}
		\# service nova-api restart \\
		\# service nova-scheduler restart \\
		\# service nova-conductor restart \\
	\end{snugshade}
\end{enumerate}

\section{Configurando o nó Compute}
Esta seção  implantação de uma rede plana simples que fornece endereços IP a instâncias via DHCP.

\subsection{Instalar componentes de rede Legacy}
\begin{itemize}
	\item Instalar o seguinte pacote
	 \begin{snugshade}
		\# apt-get install nova-network nova-api-metadata
	\end{snugshade}
\end{itemize}

\subsection{Configurando a rede Legacy}
\begin{enumerate}
	\item Editar o arquivo \emph{/etc/nova/nova.conf} e colocar os seguintes dados
	\begin{snugshade}
		[DEFAULT]\\
		$\cdots$\\
		network$\_$api$\_$class = nova.network.api.API\\
		security$\_$group$\_$api = nova \\
		firewall$\_$driver = nova.virt.libvirt.firewall.IptablesFirewallDriver\\
		network$\_$manager = nova.network.manager.FlatDHCPManager\\
		network$\_$size = 254\\
		allow$\_$same$\_$net$\_$traffic = False\\
		multi$\_$host = True
		send$\_$arp$\_$for$\_$ha = True
		share$\_$dhcp$\_$address = True
		force$\_$dhcp$\_$release = True
		flat$\_$network$\_$bridge = br100
		flat$\_$interface = INTERFACE$\_$NAME
		public$\_$interface = INTERFACE$\_$NAME
	\end{snugshade}
	
	\item Reiniciar os serviços
	\begin{snugshade}
		\# service nova-network restart
		\# service nova-api-metadata restart
	\end{snugshade}
\end{enumerate}

\section{Criando uma rede inicial}
Antes de lançar a primeira instância, deve-se criar a infra-estrutura de rede virtual necessária para que a instância irá se conectar. Esta rede normalmente fornece acesso à Internet a partir de instâncias. Pode permitir o acesso à Internet para instâncias individuais usando um IP flutuante endereço e regras do grupo de segurança adequadas. O inquilino administrador detém esta rede, pois fornece acesso à rede externa para vários inquilinos.


Esta rede tem a mesma sub-rede associado com a rede física ligada à interface externa no nó de computação.Deve ser especificado uma fatia exclusivo deste sub-rede para evitar a interferência com outros dispositivos na rede externa.

\textbf{\underline{OBS}}: Executar os comandos no nó controlador

\subsubsection{Criando a rede}
\begin{enumerate}
\item De origem as credenciais de administrador para ter acesso ao admin-somente comandos CLI:
		\begin{snugshade}
			\$ source admin-openrc.sh
		\end{snugshade}			
		
\item Criar a rede
		\begin{snugshade}
			\$ nova network-create demo-net --bridge br100 --multi-host T \textbackslash
--fixed-range-v4 NETWORK$\_$CIDR
		\end{snugshade}			
		
\item Verificar a criação da rede
		\begin{snugshade}
			\$ nova net-list
		\end{snugshade}			
\end{enumerate}
