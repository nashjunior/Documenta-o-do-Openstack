\chapter{Serviço Compute}

Usa-se o OpenStack Compute para hospedar e gerenciar sistemas de computação nas nuvens. OpenStack Compute é parte principal de um sistema IaaS (Infrastructure-as-a-Service). Os módulos principais são implementados em Phyton.

OpenStack Compute interage com o OpenStack Identity para autenticação, com o serviço OpenStack Image para disco e servidor de imagens. Acesso a imagem é limitado por projetos e usuários; parte é limitada para projeto (numero de instâncias, por exemplo). OpenStack Compute pode escalar horizontalmente os padrões de software fazer o download de imagens e executar instâncias.

\begin{tabular}{|l||c|c|}
	\hline \\
	\multicolumn{2}{c}{API} \\
	\hline \hline 
	Serviço nova-api& Aceita e reponde para o usuário final computar\\& a API de chamadas. Faz cumprir algumas políticas\\& e inicia a execução de ima instância. \\
	\hline \hline
	Servico nova-api-metadata& Aceita requisições de metadados vindos das intâncias.\\& O serviço \emph{nova-api-metadata} é geralmente utilizado\\& quando executa-se em modo multi-host a instalação \emph{nova-network}\\
	\hline \hline
\end{tabular}

\begin{tabular}{|l||c|c|}
	\hline \\
	\multicolumn{2}{c}{Nucleo Compute} \\
	\hline \hline
	Serviço nova-compute& Um daemon operário que cria e termina as instâncias\\& de máquinas virtuais através da API do hypervisor\\&. Como exemplo, libvirt para KVM ou QEMU, VMwareAPI para\\& VMware. O processo e complexo. Basicamente, o daemon aceita\\& ações vindos da fila e realiza uma série de comandos\\& de sistema assim como lançar uma instância API KVM\\& e atualiza o estados na base de dados \\
	\hline \hline
	
	Serviço nova-scheduler& Tira uma requisição da instância da máquina\\& virtual da fila e determina em qual servidor\\& compute hospedeiro ele executa \\
	\hline \hline
	
	\textbf{Modulo} nova-conductor& Media as interações entre o serviço \emph{nova-compute}\\& e o banco de dados. Elimina o acesso direto para o\\& banco de dados nas nuvens feito pelo serviço\\& \emph{nova-compute}. O módulo \emph{nova-conductor} escala horizontalmente\\&. Contudo, ele nao implanta dentro dos nós onde o s\\&erviço \emph{nova-compute} executa. \\
	\hline \hline
	
	Modulo nova-cert& Um servidor daemon que serve o serviço Nova\\& Cert. Utilizado para gerar certificados para \emph{\textbf{euca-bundle-image}} \\
	\hline
\end{tabular}

\begin{tabular}{|l||c|c|}
\hline
\multicolumn{2}{|c|}{Redes para as máquinas virtuais} \\
\hline \hline
Daemon nova-network operário& Similar ao serviço \emph{nova-compute}, aceita as\\& tarefas da rede vindos da fila e manipula a \\&rede. Realiza tarefas assim como ajusta\\& e levanta as interfaces ou muda\\& as regras IPTables \\
\hline
\end{tabular}

\begin{tabular}{|l||c|c|}
\hline
\multicolumn{2}{|c|}{Interface de console} \\
\hline \hline
daemon nova-consoleauth& Autoriza tokens para usuários que consoles\\& de proxy fornecem. Este serviço deve estar\\& executando para os consoles de proxy\\& trabalharem.\\
\hline \hline

daemon nova-novncproxy& Fornece um proxy para acessar instâncias\\& em execução através de uma conexão VNC.\\& Suporta clientes novnc browser baseado.\\
\hline
\end{tabular}

\begin{tabular}{|l||c|c|}
\hline
\multicolumn{2}{|c|}{Outros Componentes} \\
\hline \hline
A fila& Um hub central para a passagem de mensagens entre\\& daemons. Normalmente implementado com RabbitMQ \\
\hline \hline

Banco de dados SQL& Armazena maioria do tempo construção tempo de execução estados para uma infra-estrutura de nuvem, incluindo tipos de instâncias disponíveis, instâncias em uso, redes disponíveis
\end{tabular}

\section{Instalando e Configurando no nó Controllador}
\subsubsection{Configurando os pre-requisitos}
Antes de instalar e configurar o serviço Compute, deve ser criado um banco de dados, as credenciais de serviço, e ponto final API.

\begin{enumerate}
		\item Acessar o banco de dados de acesso cliente para conectar ao servidor da base de dados como \emph{root} usuário:
		\begin{snugshade}
			\$ mysql -u root -p
		\end{snugshade}
	
		\item Para criar o banco de dados, executar as seguintes instruçoes
		\begin{snugshade}
			CREATE DATABASE nova; \\ \\
			GRANT ALL PRIVILEGES ON nova.* TO 'nova'@'localhost' IDENTIFIED BY 'NOVA$\_$DBPASS'; \\ \\
			GRANT ALL PRIVILEGES ON nova.* TO 'nova'@'\%' IDENTIFIED BY '\emph{NOVA$\_$DBPASS}'
			\end{snugshade}
			
		\item De origem as credenciais de administrador para ter acesso ao admin-somente comandos CLI:
		\begin{snugshade}	
			\$ source admin-openrc.sh
		\end{snugshade}	
		
		\item Para criar as credenciais de serviço, execute as seguintes funcoes
		\begin{itemize}
			\item Criar usuario nova
				\begin{snugshade}	
					\$ 	keystone user-create --name nova --pass \$KEYSTONE$\_$USER$\_$NOVA$\_$PASSWORD --email \$EMAIL$\_$NOVA
				\end{snugshade}	
	
	
			\item Adicionar a regra \emph{admin}	para o usuário \emph{nova} e o projeto de serviço
				\begin{snugshade}	
					\$ keystone user-role-add --user nova --tenant service --role admin
				\end{snugshade}	

			\item Criar a entidade de serviço \emph{nova}
				\begin{snugshade}
						keystone service-create --name nova --type compute--description "OpenStack Compute"
				\end{snugshade}			
			
			
			\item Criar API endpoint do serviço de imagem
				\begin{snugshade}	
					\$ keystone endpoint-create \textbackslash \\
	  --service-id \$(keystone service-list \textbar awk '\textbar image / {print \$2}') \textbackslash \\
	  --publicurl http://controller:9292/v2/\%\textbackslash(tenant$\_$id\textbackslash)s  \textbackslash \\
	  --internalurl http://controller:9292/v2/\%\textbackslash(tenant$\_$id\textbackslash)s  \textbackslash \\
	  --adminurl http://controller:9292/v2/\%\textbackslash(tenant$\_$id\textbackslash)s  \textbackslash \\
	  --region regionOne
				\end{snugshade}	

		\end{itemize}				
\end{enumerate}

\subsection{Instalando e Configurando os componentes do nó controlador}
\begin{enumerate}
	\item Instalar os pacotes
	\begin{snugshade}
		\# apt-get install nova-api nova-cert nova-conductor nova-consoleauth nova-novncproxy nova-scheduler python-novaclient
	\end{snugshade}
	
	\item Editar o arquivo \emph{/etc/nova/nova.conf} com os seguintes dados
	\begin{snugshade}
		[default]\\
		$\cdots$\\
		verbose = true\\
		rpc$\_$backend = rabbit	\\
		auth\_strategy = keystone\\
		my$\_$ip= 10.129.64.51
		vncserver$\_$listen = 10.129.64.51
		vncserver$\_$proxyclient$\_$address = 10.129.64.51
		
		$\cdots$\\ \\
		
		[database] \\
		$\cdots$ \\
		connection = mysql://glance:NOVA$\_$\emph{DBPASS@}controller/glance \\ \\
		
		$\cdots$\\ \\
		
		[oslo$\_$messaging$\_$rabbit]\\
		$\cdots$
		rabbit$\_$host = controller \\
		rabbit$\_$userid = openstack \\
		rabbit$\_$password = RABBIT$\_$PASS\\
		
		$\cdots$\\ \\		
		
		[keystone$\_$authtoken] \\
		$\cdots$ \\
		auth$\_$uri = http://controller:5000 \\
		auth$\_$url = http://controller:35357 \\
		auth$\_$plugin = password \\
		project$\_$domain$\_$id = default \\
		user$\_$domain$\_$id = default \\
		project$\_$name = service \\				
		username = glance \\
		password = \emph{•}{NOVAPASS}\\ \\
		$\cdots$ \\ \\
						
		[glance]\\
		$\cdots$		\\
		host = \emph{controller} \\ \\
		$\cdots$ \\ \\		
		
		[oslo$\_$concurrency] \\
		$\cdots$\\
		lock$\_$path = /var/lib/nova/tmp
	\end{snugshade}
	
	\item Popular a base de dados do Compute
	\begin{snugshade}
		\# su -s /bin/sh -c "nova-manage db sync" nova
	\end{snugshade}
\end{enumerate}

\subsection{Finalizando a instalação}
\begin{enumerate}
	\item Reiniciar os serviços Compute
	\begin{snugshade}
		\# service nova-api restart
		\# service nova-cert restart
		\# service nova-consoleauth restart
		\# service nova-scheduler restart
		\# service nova-conductor restart
		\# service	nova-novncproxy restart
	\end{snugshade}
	
	\item Por padrão, os pacotes Ubuntu cria um banco de dados SQLite. 
	Por conta desta configuracao utilizar o servidor banco de dados SQL, remover a base de dados do arquivo SQLite:
	\begin{snugshade}
		\# rm -f /var/lib/nova/nova.db
	\end{snugshade}		
\end{enumerate}

\section{Instalando e Configurando no nó Compute}

Instalando e configurando os componentes do hypervisor no Compute
\begin{enumerate}
	\item Instalar os pacotes
	\begin{snugshade}
		\# apt-get install nova-compute sysfsutils
	\end{snugshade}	
	
	\item Editar o arquivo \emph{/etc/nova/nova.conf} e inserir os seguintes dados
	\begin{snugshade}
		[default]\\
		$\cdots$\\
		verbose = true\\
		rpc$\_$backend = rabbit	\\
		auth$\_$estrategy = keystone\\
		my$\_$ip= 	10.129.64.61
		vnc$\_$enabled = true
		vncserver$\_$listen = 0.0.0.0
		vncserver$\_$proxyclient$\_$address = 10.129.64.61
		novncproxy$\_$base$\_$url = http://controller:6080/vnc$\_$auto.html
		
		$\cdots$\\ \\
		
		[database] \\
		$\cdots$ \\
		connection = mysql://glance:NOVA$\_$\emph{DBPASS@}controller/glance \\ \\
		
		$\cdots$\\ \\
		
		[oslo$\_$messaging$\_$rabbit]\\
		$\cdots$
		rabbit$\_$host = controller \\
		rabbit$\_$userid = openstack \\
		rabbit$\_$password = RABBIT$\_$PASS\\
		
		$\cdots$\\ \\		
		
		[keystone$\_$authtoken] \\
		$\cdots$ \\
		auth$\_$uri = http://controller:5000 \\
		auth$\_$url = http://controller:35357 \\
		auth$\_$plugin = password \\
		project$\_$domain$\_$id = default \\
		user$\_$domain$\_$id = default \\
		project$\_$name = service \\				
		username = glance \\
		password = \emph{NOVAPASS}\\ \\
		$\cdots$ \\ \\

		[glance]\\
		$\cdots$		\\
		host = \emph{controller} \\ \\
		$\cdots$ \\ \\		
		
		[oslo$\_$concurrency] \\
		$\cdots$\\
		lock$\_$path = /var/lib/nova/tmp		
	\end{snugshade}
\end{enumerate}

\subsection{Finalizando a instalação}
\begin{enumerate}
	\item Determinar se o nó compute suporta aceleração de hardware para maquinas virtuais
	\begin{snugshade}
		\$ egrep -c '(vmx|svm)' /proc/cpuinfo
	\end{snugshade}
	
	Se o comando retornar valor maior ou igual a 1, o nó compute suporta aceleração de hardware.
	Se o comando retornar valor igual a 0, o nó compute nao suporta aceleração de hardware e é então necessário configurar \emph{libvirt} para usar QEMU no lugar de KVM
	
	\begin{itemize}
	 \item	Edite a seção \emph{[libvirt]} em \emph{/etc/nova/nova-compute.conf} como mostra a seguir
		\begin{snugshade}
			[libvirt]\\
			$\cdots$\\
			virt$\_$type=qemu
		\end{snugshade}
	\end{itemize}
	
	\item Reiniciar o serviço Compute
	\begin{snugshade}
		\# service nova-compute restart
	\end{snugshade}
	
	\item Por padrão, os pacotes Ubuntu cria um banco de dados SQLite. 
	Por conta desta configuracao utilizar o servidor banco de dados SQL, remover a base de dados do arquivo SQLite:
	\begin{snugshade}
		\# rm -f /var/lib/nova/nova.db
	\end{snugshade}		
\end{enumerate}

\section{Verificar a Operação}

\begin{enumerate}
		\item De origem as credenciais de administrador para ter acesso ao admin-somente comandos CLI:
		\begin{snugshade}
			\$ source admin-openrc.sh
		\end{snugshade}			
		
		\item 	Listar os componentes de serviço para verificar se executou com sucesso e registrar cada processo
		\begin{snugshade}
			\$ nova service-list
		\end{snugshade} 
		
		\item Listar os nós extremos da API 	no serviço de autenticação para verificar conectividade com o serviço de autenticação
		\begin{snugshade}
			\$ nova endpoints
		\end{snugshade} 		
		
		\item Listar as imagens 	no catálogo de serviço de imagens para verificar a conectividade com o serviço de imagens
		\begin{snugshade}
			\$ nova image-list
		\end{snugshade} 			
\end{enumerate}